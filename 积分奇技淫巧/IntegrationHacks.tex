\documentclass[lang=cn,newtx,10pt,scheme=chinese]{elegantbook}

\title{积分的奇技淫巧}
\subtitle{Integration Hacks}

\author{Monika}
% \institute{Elegant\LaTeX{} Program}
\date{\today}
\version{0.1}
% \bioinfo{自定义}{信息}

% \extrainfo{注意:本模板自 2023 年 1 月 1 日开始,不再更新和维护!}

\setcounter{tocdepth}{3}

\logo{logo-blue.png}
\cover{cover.jpg}

% 本文档命令
\usepackage{array}
\newcommand{\ccr}[1]{\makecell{{\color{#1}\rule{1cm}{1cm}}}}

\renewcommand{\textbf}[1]{\text{\heiti #1}}

% 修改标题页的橙色带
\definecolor{customcolor}{RGB}{32,178,170}
\colorlet{coverlinecolor}{customcolor}
\usepackage{cprotect}
\usepackage{tikz}
\usepackage{float}
\usepackage{soul}
\usepackage{arcs}
\usetikzlibrary{arrows.meta,3d}

\addbibresource[location=local]{reference.bib} % 参考文献,不要删除
\DeclareMathOperator{\arsinh}{arsinh}
\DeclareMathOperator{\arcosh}{arcosh}
\DeclareMathOperator{\artanh}{artanh}
\DeclareMathOperator{\arsech}{arsech}
\DeclareMathOperator{\arcoth}{arcoth}
\DeclareMathOperator{\arcsch}{arcsch}
\definecolor{lightyellow}{rgb}{1, 1, 0.8} % 定义一个淡黄色
\sethlcolor{lightyellow} % 设置为高亮颜色
\begin{document}

\maketitle
\frontmatter

\tableofcontents

\chapter{绪论}
本书是是 LZU 数学协会举办的数学讲座的讲义,将会讲授一些在课不上不会讲授,但在某些情况有奇效的积分技巧,近年来 cmc 的试题中偶尔会出现此类题目

笔者水平有限,若有不足缺漏之处恳请读者更正,或在 \href{https://github.com/Monika-shipship/MonikaMathLecture}{Github仓库处(点击即可跳转)} 提交PR以更正
\mainmatter

\chapter{基础知识}
在本章列举出关于微积分最基础的知识,读者应做到在看到本章的所有题目时能瞬间反应出答案来
\section{极限}
所谓的极限,就是 Approach ,不断地逼进一个值,但是始终不会到达

用数学语言来讲就是
\begin{equation}
  \lim_{x \to a} f(x)=A \iff \forall \epsilon>0,\exists \delta>0, \text{使得当}x \in \left[ a-\epsilon,a+\epsilon \right], \left\vert f(x) -A\right\vert <\epsilon
\end{equation}
如果函数是多元函数,则定义为从任意路径趋近于该点的值都相同,注意是任意路径而不是任意角度,取所有斜率的直线并不能穷尽所有路径,比如下面这个
\begin{problem}
    $\lim_{(x,y) \to (0,0)} \frac{xy}{x+y} $ 是否存在?
  \end{problem}
  如果你取所有斜率的直线,比如 $y=kx$,而误以为直线能穷尽所有路径,就会得出极限为零的错误答案 $\lim_{(x,y) \to (0,0)} f(x,y)=\lim \frac{(kx^{2})}{(1+k)x}=\lim_{x \to 0} \frac{k}{(1+k)}x=0$
  \begin{solution}
    不存在,取路径 $y=-x+x^{2},\lim_{(x,y) \to (0,0)} \frac{x(-x-x^{2})}{x^{2}}=\lim_{x \to 0}(-1-x)=-1$\
    这条路径得到的值和 $y=kx$ 路径得到的值不同,所以极限不存在
  \end{solution}
\section{导数}
我们使用最朴素的导数理解方式,自变量发生微小变化后,因变量会随之发生一个微小变化,这两个变化的比值就是导数,也是斜率

\begin{equation}
  \frac{\mathrm{d}y}{\mathrm{d}x} \big|_{x_0}\equiv \lim_{\Delta x \to 0} \frac{\Delta y}{\Delta x}=\lim_{x \to x_0} \frac{y(x)-y(x_0)}{x-x_0}
\end{equation}
利用这个定义不难算出一些初等函数的导数

另外需要注意的是,在最开始学习微积分时,最好不要使用 $y^{\prime }$ 这种记号来表示一阶导,因为这不利于你最开始的理解导数的计算法则,不过当你熟悉之后,随便怎么用都行,在开始阶段,最好使用 $\frac{\mathrm{d}y}{\mathrm{d}x}$ 这种记号,并且直接将 $\mathrm{d}y,\mathrm{d}x$ 当成普通的数来运算,可加可减可乘可除,这虽然略失严谨,但能帮助你快速入门
\subsubsection{求导的法则}


要得到;这个法则非常简单,我们将 $\Delta x$ 视为一个有限的小量,而 $\mathrm{d}x$ 是将这个小量趋于零,所以每次你见到 $\mathrm{d}$ ,就已经暗示了这里有一个量会趋近于零 

\begin{itemize}

  \item 乘法,前导加后导
  
  $\Delta(uv)=(u+\Delta u)(v+\Delta v)-uv=uv+u \Delta v+v \Delta u+ \Delta u \Delta v - uv =u \Delta v+v \Delta u+ \Delta u \Delta v$
  
  $\Delta u \Delta v$ 是两个微小量相乘,比一个微小量更小,认为他是高阶小量,有几个微小量相乘就是几阶小量,这里 $\Delta u \Delta v$ 就是二阶小量,他和一阶小量 $v \Delta u+ \Delta u$ 相比可以略去,于是得到 $\Delta(uv)\approx u \Delta v+v \Delta u $ 注意只有这一步才是约等于,因为略去了二阶小量

  当我们取 $\Delta u ,\Delta v $ 都趋近于零的时候,约等号就变成了等号,同时也把 $\Delta$ 换成 $\mathrm{d}$

  于是我们说 $\mathrm{d} (uv)=u \mathrm{d} v+v\mathrm{d}u$ 我们管这个叫 \textcolor{red}{前导加后导}

  上面阐释了求导数的基本方法,再用相同的方法求除法

  \item 除法,上导减下导
  
  $\Delta \left( \frac{u}{v} \right) = \frac{u+\Delta u}{v+\Delta v}-\frac{u}{v}=\frac{uv+v \Delta u-uv-u \Delta v}{v(v+\Delta v)}=\frac{v \Delta u - u \Delta v}{v(v+\Delta v)}$

  取极限 ${\Delta v,\Delta u \to 0}$

  $d(\frac{u}{v})=\frac{v\mathrm{d}u-u\mathrm{d}v}{v^{2}}$
  
          \item 求导是线性的 $\mathrm{d}(u+v)=\mathrm{d}u+\mathrm{d}v\implies \frac{\mathrm{d}(u+v)}{\mathrm{d}t}=\frac{\mathrm{d}u}{\mathrm{d}t}+\frac{\mathrm{d}v}{\mathrm{d}t}$
          \item 链式法则,这一点将 $\mathrm{d}x$ 这种量整体看作是普通的数计算即可(说是整体是你不要干出 $\frac{dy}{dx}$ 把 d 给约去了这种事来),于是所谓链式就很简单了
          
          以下是帮助你快速理解的,而不是证明

          $\frac{\mathrm{d}y}{\mathrm{d}t}$ 这个式子乘以 $\frac{\mathrm{d}x}{\mathrm{d}x}=1$ 后得到 $\frac{\mathrm{d}y}{\mathrm{d}t} \cdot \frac{\mathrm{d}x}{\mathrm{d}x}=\frac{\mathrm{d}y}{\mathrm{d}x}\cdot \frac{\mathrm{d}x}{\mathrm{d}t}$
          \begin{example}
            \begin{equation}
              \begin{aligned}
              (e^{\sqrt{\sin ^{2}x+\cos x}})'&=\frac{\mathrm{d}(e^{\sqrt{\sin ^{2}x+\cos x}})}{\mathrm{d}\sqrt{\sin ^{2}x+\cos x}}\cdot \frac{\mathrm{d}\sqrt{\sin ^{2}x+\cos x}}{\mathrm{d}x} \\
              &=e^{\sqrt{\sin ^{2}x+\cos x}}\cdot \frac{\mathrm{d}\sqrt{\sin ^{2}x+\cos x}}{\mathrm{d}\sin ^{2}x+\cos x}\cdot \frac{\mathrm{d}(\sin ^{2}x+\cos x)}{\mathrm{d}x}\\
              &=e^{\sqrt{\sin ^{2}x+\cos x}}\cdot \frac{1}{2 \sqrt{(\sin ^{2}x+\cos x)}}\cdot (2 \sin x \cos x - \sin x )
              \end{aligned}
            \end{equation}
            于是每一步,只要将其中的一大坨变量看作整体,就能机械地进行计算了,比如第一步就是将 $\sqrt{\sin ^{2}x+\cos x}=t$ 看作整体, $\frac{\mathrm{d}(e^{\sqrt{\sin ^{2}x+\cos x}})}{\mathrm{d}\sqrt{\sin ^{2}x+\cos x}}$ 自然就是 $\frac{\mathrm{d}e^{t}}{\mathrm{d}t}=e^{t}$
          \end{example}
          \item 反函数
          
          将 $\mathrm{d}y,\mathrm{d}x$ 看作数,立刻就有
          \begin{equation}
            \frac{\mathrm{d}y}{\mathrm{d}x}=\frac{1}{\frac{\mathrm{d}x}{\mathrm{d}y}}
          \end{equation}
          即若 $y=f(x)$ 的反函数是 $x=g(y)$
        ,则
        \begin{equation}
          \frac{\mathrm{d}y}{\mathrm{d}x}=f^{\prime }(x)=\frac{1}{\frac{\mathrm{d}x}{\mathrm{d}y}}=\frac{1}{g^{\prime }(y)}
        \end{equation}
        \item 对于多元函数,比如 $z=f(x(t),y(t))$ ,现在想求 $z$ 对 $t$ 的导数,从直觉上讲,当 $t$ 有一微小变化 $\mathrm{d}t$ 后, $x,y$ 也会有相应的变化 $\mathrm{d}x,\mathrm{d}y$ ,然后这个 $\mathrm{d}x,\mathrm{d}y$ 再影响 $z$ 的变化
        
        如何具体求出呢?你可以先令 $y$ 不变,那么 $\mathrm{d}y=0$ ,再看看 $\mathrm{d}x$ 会怎么影响 $z$

        于是 $z(x+\mathrm{d}x,y)-z(x,y)=\mathrm{d}z|_{\mathrm{d}y=0}=\mathrm{d}f=\frac{\mathrm{d}f}{\mathrm{d}x}\mathrm{d}x$

        

        在上面 $x$ 变化的基础上,再让 $y$ 变化 $\mathrm{d}y$ 
        
        \begin{equation}
          \begin{aligned}
          \mathrm{d}z&=z(x+\mathrm{d}x,y+\mathrm{d}y)-z(x,y)\\
          &=z(x+\mathrm{d}x,y+\mathrm{d}y)-z(x,y+\mathrm{d}y)+z(x,y+\mathrm{d}y)-z(x,y) \\
          &=\frac{\mathrm{d}f}{\mathrm{d}x}|_{(x,y+\mathrm{d}y)} \mathrm{d}x+\frac{\mathrm{d}f}{\mathrm{d}y}|_{(x,y)} \mathrm{d}y\\
          &=\frac{\mathrm{d}f}{\mathrm{d}x}|_{(x,y)} \mathrm{d}x+\frac{\mathrm{d}f}{\mathrm{d}y}|_{(x,y)} \mathrm{d}y\\
          \end{aligned}
        \end{equation}
如果按照高数老师们的习惯,将 $\frac{\mathrm{d}f}{\mathrm{d}x}$ 记作 $f^{\prime }_{x} $ (其中有几个撇代表是几阶导,而下标是对谁求导,也有人不写撇)则是
        $\mathrm{d}z=f^{\prime }_{x}\mathrm{d}x+f^{\prime }_{y}\mathrm{d}y$

如果 $z$ 对 $t$ 求导就是在上式左右两边同时除 $\mathrm{d}t$ ,就得到 $\frac{\mathrm{d}z}{\mathrm{d}t}=f^{\prime }_{x}\frac{\mathrm{d}x}{\mathrm{d}t}+f^{\prime }_{y}\frac{\mathrm{d}y}{\mathrm{d}t}$

如果使用下标来代表对谁求导,则是

\begin{equation}
  z^{\prime }_{t}=f^{\prime }_{x}x^{\prime }_{t}+f^{\prime }_{y}y^{\prime }_{t}
\end{equation}
\end{itemize}

\subsubsection{具体求导例子}

我们已经完全了解了抽象的求导的方法,下面再来具体算一些函数

当然我们能算出结果来的,一般都是初等函数

数学课上会告诉你严格的定义是

\begin{definition}{初等函数}
  所谓初等函数就是幂函数,指数函数,对数函数,三角函数,反三角函数通过有限次的加减乘除,有限次的复合运算所得的函数
\end{definition}

对我们而言,你自高中起就熟悉的各种函数的有限次组合,就叫初等函数

以 $x^{n}$ 和 $\sin x,\cos x,e^{x}$ 的导数为例,具体推导其导数形式
\begin{example}
  \begin{equation}
    \begin{aligned}
      (x^{n})^{\prime }&=\lim_{\Delta x \to 0}\frac{(x+\Delta x)^{n}-x^{n}}{\Delta x} \\
      &=\lim_{\Delta x \to 0}\frac{(x^{n}+nx^{n-1}\Delta x+o(\Delta x))-x^{n}}{\Delta x} \\
      &=\lim_{\Delta x \to 0}\frac{nx^{n-1}\Delta x+o(\Delta x)}{\Delta x} \\
      &=nx^{n-1}
    \end{aligned}
  \end{equation}
\end{example}
\begin{example}
  \begin{equation}
    \begin{aligned}
      (\sin x)^{\prime } &= \lim_{\Delta x \to 0} \frac{\sin(x+\Delta x) - \sin x}{\Delta x} \\
      &= \lim_{\Delta x \to 0} \frac{\sin x \cos \Delta x + \cos x \sin \Delta x - \sin x}{\Delta x} \\
      &= \lim_{\Delta x \to 0} \frac{\sin x (\cos \Delta x - 1) + \cos x \sin \Delta x}{\Delta x} \\
      &= \lim_{\Delta x \to 0} \left( \sin x \cdot \frac{\cos \Delta x - 1}{\Delta x} + \cos x \cdot \frac{\sin \Delta x}{\Delta x} \right) \\
      &= \sin x \cdot 0 + \cos x \cdot 1 \\
      &= \cos x
    \end{aligned}
  \end{equation}
  \begin{equation}
    \begin{aligned}
      (\cos x)^{\prime } &= \lim_{\Delta x \to 0} \frac{\cos(x+\Delta x) - \cos x}{\Delta x} \\
      &= \lim_{\Delta x \to 0} \frac{\cos x \cos \Delta x - \sin x \sin \Delta x - \cos x}{\Delta x} \\
      &= \lim_{\Delta x \to 0} \frac{\cos x (\cos \Delta x - 1) - \sin x \sin \Delta x}{\Delta x} \\
      &= \lim_{\Delta x \to 0} \left( \cos x \cdot \frac{\cos \Delta x - 1}{\Delta x} - \sin x \cdot \frac{\sin \Delta x}{\Delta x} \right) \\
      &= \cos x \cdot 0 - \sin x \cdot 1 \\
      &= -\sin x
    \end{aligned}
  \end{equation}
\end{example}
\begin{example}
  对于指数函数 $e^x$ 的导数,我们首先需要引入一个与常数 $e$ 的定义相关的重要极限,关于这个重要极限的证明我们放在文末 \ref{prf:ImtLim}
  \begin{equation}
    \lim_{h \to 0} \frac{e^h - 1}{h} = 1
  \end{equation}
  这个极限的几何意义是函数 $y=e^x$ 在点 $(0,1)$ 处的切线斜率为 1。基于这个基本极限,我们可以推导 $e^x$ 在任意点 $x$ 处的导数。
  \begin{equation}
    \begin{aligned}
      (e^x)' &= \lim_{\Delta x \to 0} \frac{e^{x+\Delta x} - e^x}{\Delta x} \\
      &= \lim_{\Delta x \to 0} \frac{e^x e^{\Delta x} - e^x}{\Delta x} \\
      &= e^x \lim_{\Delta x \to 0} \frac{e^{\Delta x} - 1}{\Delta x} \\
      &= e^x \cdot 1 \\
      &= e^x
    \end{aligned}
  \end{equation}
  这个结果表明,$e^x$ 是一个导数等于其自身的函数

  如果你将求导看作是一个函数,或者说算子 $D$ ,那么 $D(e^{x} )=e^{x}  $ ,也可称 $e^{x}$ 是求导的不动点(不动点的定义就是 $f(x)=x$的解),这个想法在日后用微分算子法快速解微分方程时有用
\end{example}
\begin{example}
  下面求对数函数的导数。
  我们首先求自然对数函数 $y = \ln x$ 的导数。根据定义,$\ln x$ 是指数函数 $e^x$ 的反函数。
  \begin{equation}
    y = \ln x \iff x = e^y
  \end{equation}
  我们可以利用反函数求导法则。对方程 $x = e^y$ 两边同时对 $x$ 求导:
  \begin{equation}
    \begin{aligned}
      \frac{d}{dx}(x) &= \frac{d}{dx}(e^y) \\
      1 &= e^y \cdot \frac{dy}{dx} \quad (\text{根据链式法则})
    \end{aligned}
  \end{equation}
  从中解出 $\frac{dy}{dx}$:
  \begin{equation}
    \frac{dy}{dx} = \frac{1}{e^y}
  \end{equation}
  因为 $x = e^y$,所以我们得到:
  \begin{equation}
    (\ln x)' = \frac{1}{x}
  \end{equation}
  对于更一般的对数函数 $y = \log_a x$,我们可以使用换底公式将其转换为自然对数:
  \begin{equation}
    y = \log_a x = \frac{\ln x}{\ln a}
  \end{equation}
  由于 $\frac{1}{\ln a}$ 是一个常数,所以:
  \begin{equation}
    (\log_a x)' = \frac{d}{dx}\left(\frac{\ln x}{\ln a}\right) = \frac{1}{\ln a} \cdot (\ln x)' = \frac{1}{x \ln a}
  \end{equation}
\end{example}

\begin{example}
  反三角函数的导数,如 $\arcsin x,\arccos x,\arctan x$,可以使用反函数求导法则简便地求出。
  \begin{itemize}
    \item 求 $(\arcsin x)'$:
    设 $y = \arcsin x$,则 $x = \sin y$,其中 $y \in [-\frac{\pi}{2}, \frac{\pi}{2}]$。
    根据反函数求导法则,$(\arcsin x)' = \frac{dy}{dx} = \frac{1}{\frac{dx}{dy}} = \frac{1}{(\sin y)'} = \frac{1}{\cos y}$。
    由于 $y \in [-\frac{\pi}{2}, \frac{\pi}{2}]$,$\cos y \ge 0$,因此 $\cos y = \sqrt{1-\sin^2 y} = \sqrt{1-x^2}$。
    所以,$\displaystyle (\arcsin x)' = \frac{1}{\sqrt{1-x^2}}$。

    \item 求 $(\arccos x)'$:
    设 $y = \arccos x$,则 $x = \cos y$,其中 $y \in [0, \pi]$。
    $(\arccos x)' = \frac{1}{(\cos y)'} = \frac{1}{-\sin y}$。
    由于 $y \in [0, \pi]$,$\sin y \ge 0$,因此 $\sin y = \sqrt{1-\cos^2 y} = \sqrt{1-x^2}$。
    所以,$\displaystyle (\arccos x)' = -\frac{1}{\sqrt{1-x^2}}$。

    \item 求 $(\arctan x)'$:
    设 $y = \arctan x$,则 $x = \tan y$,其中 $y \in (-\frac{\pi}{2}, \frac{\pi}{2})$。
    $(\arctan x)' = \frac{1}{(\tan y)'} = \frac{1}{\sec^2 y}$。
    利用恒等式 $\sec^2 y = 1 + \tan^2 y = 1+x^2$。
    所以,$\displaystyle (\arctan x)' = \frac{1}{1+x^2}$。
  \end{itemize}
  一个有趣的关系是 $(\arcsin x)' + (\arccos x)' = 0$,这与恒等式 $\arcsin x + \arccos x = \frac{\pi}{2}$ 两边求导的结果是一致的。
\end{example}
\begin{example}
  双曲函数虽然名字里带个“双曲”,听起来很吓人,但它们本质上就是指数函数 $e^x$ 的简单组合。它们与三角函数有很多相似的性质,因此学起来并不难。
  \subsubsection{双曲函数定义}
  最基本的两个双曲函数是双曲正弦(sinh)和双曲余弦(cosh):
  \begin{align}
    \sinh x &= \frac{e^x - e^{-x}}{2} \\
    \cosh x &= \frac{e^x + e^{-x}}{2}
  \end{align}
  其他的双曲函数也和三角函数类似,由这两个基本函数组合而成:
  \begin{equation}
    \tanh x = \frac{\sinh x}{\cosh x}, \quad \coth x = \frac{1}{\tanh x}, \quad \text{sech}\,x = \frac{1}{\cosh x}, \quad \text{csch}\,x = \frac{1}{\sinh x}
  \end{equation}
  
  \subsubsection{性质与三角函数的深刻联系}
  为什么叫它们“双曲”函数呢?因为它们和双曲线有关。我们知道三角函数 $(\cos t, \sin t)$ 构成了单位圆 $x^2+y^2=1$ 上的点,而双曲函数 $(\cosh t, \sinh t)$ 则构成了单位双曲线 $x^2-y^2=1$ 上的点。
  我们可以很容易地验证这个核心性质:
  \begin{equation}
    \begin{aligned}
      \cosh^2 x - \sinh^2 x &= \left(\frac{e^x + e^{-x}}{2}\right)^2 - \left(\frac{e^x - e^{-x}}{2}\right)^2 \\
      &= \frac{1}{4} \left( (e^{2x} + 2 + e^{-2x}) - (e^{2x} - 2 + e^{-2x}) \right) \\
      &= \frac{1}{4}(4) = 1
    \end{aligned}
  \end{equation}
  这个恒等式 $\cosh^2 x - \sinh^2 x = 1$ 是双曲函数最重要的性质。但它和三角函数的 $\cos^2 x + \sin^2 x = 1$ 之间有什么联系呢?答案藏在复数域里。
  
  根据欧拉公式 $e^{i\theta} = \cos\theta + i\sin\theta$,我们可以得到:
  \begin{align}
    \cos\theta &= \frac{e^{i\theta} + e^{-i\theta}}{2} \\
    \sin\theta &= \frac{e^{i\theta} - e^{-i\theta}}{2i}
  \end{align}
  现在,我们把自变量从实数 $x$ 换成纯虚数 $ix$:
  \begin{align}
    \cosh(ix) &= \frac{e^{ix} + e^{-ix}}{2} = \cos x \\
    \sinh(ix) &= \frac{e^{ix} - e^{-ix}}{2} = i \cdot \frac{e^{ix} - e^{-ix}}{2i} = i\sin x
  \end{align}
  反过来,我们也可以得到:
  \begin{align}
    \cos(ix) &= \frac{e^{i(ix)} + e^{-i(ix)}}{2} = \frac{e^{-x} + e^{x}}{2} = \cosh x \\
    \sin(ix) &= \frac{e^{i(ix)} - e^{-i(ix)}}{2i} = \frac{e^{-x} - e^{x}}{2i} = \frac{-(e^x - e^{-x})}{2i} = i \cdot \frac{e^x - e^{-x}}{2} = i\sinh x
  \end{align}
  这两个关系 $\cos(ix) = \cosh x$ 和 $\sin(ix) = i\sinh x$ 是连接三角函数与双曲函数的桥梁。利用它们,我们可以把任何一个三角函数的恒等式“翻译”成双曲函数的恒等式。
  
  例如,我们从 $\cos^2 x + \sin^2 x = 1$ 出发,将 $x$ 替换为 $ix$:
  \begin{equation}
    \cos^2(ix) + \sin^2(ix) = 1 \implies (\cosh x)^2 + (i\sinh x)^2 = 1 \implies \cosh^2 x - \sinh^2 x = 1
  \end{equation}
  这样,我们得到了双曲函数的基本恒等式。
  
  再比如和角公式 $\cos(x+y) = \cos x \cos y - \sin x \sin y$,我们把 $x, y$ 分别换成 $ix, iy$:
  \begin{align*}
    \cos(ix+iy) &= \cos(ix)\cos(iy) - \sin(ix)\sin(iy) \\
    \cosh(x+y) &= (\cosh x)(\cosh y) - (i\sinh x)(i\sinh y) \\
    \cosh(x+y) &= \cosh x \cosh y + \sinh x \sinh y
  \end{align*}
  这提供了一个从三角函数公式推导双曲函数公式的绝佳方法,比死记硬背优雅得多。
  
  \paragraph{快速翻译法则}
  用三角恒等式快速“翻译”出双曲恒等式的经验法则:
  \begin{itemize}
    \item 将 $\sin\,\to\,\sinh$,$\cos\,\to\,\cosh$;
    \item 若出现“两个 $\sinh$ 相乘”的项(如 $\sinh^2x$、$\sinh x\,\sinh y$),则把该项的符号取反
    
  \end{itemize}

  实际上就是
  \begin{equation}
    \begin{cases} \sin ix\to  i \sinh x     , &   \\\cos ix\to  \cosh x , & \\ \tan ix =i \tanh x  \end{cases}
  \end{equation}

  例如:$\cos(x+y)=\cos x\cos y-\sin x\sin y$ 按规则得到 $\cosh(x+y)=\cosh x\cosh y+\sinh x\sinh y$。

  \paragraph{三角-双曲性质对比表格}
  \begin{table}[H]
    \centering
    \begin{tabular}{l|l}
      \hline
      三角函数 & 双曲函数 \\
      \hline
      $\cos^2x+\sin^2x=1$ & $\cosh^2x-\sinh^2x=1$ \\
      $\cos(x\!\pm\!y)=\cos x\cos y\mp\sin x\sin y$ & $\cosh(x\!\pm\!y)=\cosh x\cosh y\pm\sinh x\sinh y$ \\
      $\sin(x\!\pm\!y)=\sin x\cos y\pm\cos x\sin y$ & $\sinh(x\!\pm\!y)=\sinh x\cosh y\pm\cosh x\sinh y$ \\
      $\cos 2x=\cos^2x-\sin^2x$ & $\cosh 2x=\cosh^2x+\sinh^2x$ \\
      $\sin 2x=2\sin x\cos x$ & $\sinh 2x=2\sinh x\cosh x$ \\
      $(\cos x)'=-\sin x$ & $(\cosh x)'=\sinh x$ \\
      $(\sin x)'=\cos x$ & $(\sinh x)'=\cosh x$ \\
      \hline
    \end{tabular}
  \end{table}

  \paragraph{倍角公式(显式写出)}
  由和角公式取 $y=x$(或直接由指数定义)可得:
  \begin{align}
    \cosh(2x) &= \cosh^2 x + \sinh^2 x=1+2 \sinh^2 x=2 \cosh^2 x-1, \\
    \sinh(2x) &= 2\sinh x\,\cosh x, \\
    \tanh(2x) &= \frac{2\tanh x}{1+\tanh^2 x} \;=\; \frac{2\sinh x\,\cosh x}{\cosh^2 x+\sinh^2 x}.
  \end{align}
  说这么多,到底是用来干什么的呢?在积分的时候,他换元很有用,比如 $\sqrt{1+x^{2}}$ 的积分用双曲函数换元就会方便很多, $\sinh \theta= x,\cosh \theta=\sqrt{1+x^{2}}$ 
  
  \begin{equation}
    \begin{aligned}
    \int \sqrt{1+x^{2}} \mathrm{d}x & =\int \cosh \theta \mathrm{d}(\sinh \theta)\\
    &=\int \cosh ^{2}\theta \mathrm{d}\theta\\
    &=\int \frac{(1+ \cosh 2\theta)}{2} \mathrm{d}\theta \\
    &=\frac{1}{2} \theta +\frac{1}{4} \sinh 2\theta+c\\
    &=\frac{1}{2}  \arcosh x+\frac{1}{4} 2 \sinh \theta \cosh \theta\\
    &=\frac{1}{2} \ln(x+\sqrt{1+x^{2}})+\frac{1}{2}x \sqrt{1+x^{2}}+C
    \end{aligned}
  \end{equation}
\end{example}


\begin{example}
  
  反双曲函数就是把 $\sinh,\cosh,\tanh$ 那些拿来求“反解”。比如 $\sinh \theta = x$ 想要 $\theta$,我们就写成 $\theta=\arsinh x$。从图像上看,$\arsinh$ 是 $\sinh$ 的“左右翻转”,定义域是全体实数,值域也还是全体实数;$\arcosh$ 则只能从 $\cosh x \ge 1$ 的那一截翻上去,值域限制在 $[0,+\infty)$。$\artanh$ 对应的是 $\tanh$ 在 $(-1,1)$ 上的单调部分。
  
  用指数函数去解方程可以得到它们和对数的关系:
  \begin{align}
    \arsinh x &= \ln\!\left(x + \sqrt{x^{2}+1}\right) \qquad &&(x \in \mathbb{R}), \\
    \arcosh x &= \ln\!\left(x + \sqrt{x^{2}-1}\right) \qquad &&(x \ge 1), \\
    \artanh x &= \frac{1}{2}\ln\!\frac{1+x}{1-x} \qquad &&(|x|<1).
  \end{align}
  以 $\arsinh$ 为例,把“反解”的步骤详细写出来就是:
  \begin{align*}
    y &= \arsinh x \quad \Longleftrightarrow \quad \sinh y = x, \\
    \sinh y &= \frac{e^{y}-e^{-y}}{2} = x, \\
    e^{2y} - 2xe^{y} - 1 &= 0 \quad \text{(把等式两边乘 $e^{y}$,凑成一元二次)}, \\
    e^{y} &= x + \sqrt{x^{2}+1} \quad \text{(只取正根,因为 $e^{y}>0$)}, \\
    y &= \ln\!\left(x+\sqrt{x^{2}+1}\right).
  \end{align*}
  看清这一步后,$\arcosh$、$\artanh$ 完全一样照做,只是换成各自的指数表达式。
  这些公式的推导和三角反函数类似:先把定义式写成指数形式,再凑成一个对数。它们一方面告诉我们反双曲函数可以用初等函数表示,另一方面也便于做极限或者积分。
  
  还记得前面那句 $\sin(ix)=i\sinh x$ 吗?可以不用任何对数公式,直接由“反函数”的唯一性得到关系式。令
  \begin{equation}
    \begin{aligned}
    \sin i x= i \sinh x \\
    \sin (\arcsin ix)=ix\\
    \sin(i(-i)\arcsin ix)=ix\\
    i \sinh ((-i)\arcsin ix)=i x\\
    \sinh ((-i)\arcsin ix)=x\\
    \sinh(\arsinh x)=x\\
    \implies (-i)\arcsin ix=\arsinh x\\
    \implies \textcolor{red}{\arcsin ix=i\,\arsinh x.}
    \end{aligned}
  \end{equation}
  同理有(主值,$x\in\mathbb{R}$)
  \begin{equation}
    \begin{cases}
      \arcsin (i x)= i\,\arsinh x, \\
      \arctan (i x)= i\,\artanh x, \\
      \arccos (i x)= \dfrac{\pi}{2} - i\,\arsinh x.
    \end{cases}
  \end{equation}
  其中第三条也可以和 $\arcosh$ 发生联系。注意到对任意实数 $x$,有
  \begin{equation*}
    \arcosh\!\big(\sqrt{1+x^{2}}\big) = \ln\!\big(\sqrt{1+x^{2}}+|x|\big) = |\arsinh x|,
  \end{equation*}
  因而可写成
  \begin{equation*}
    \arccos(i x)=\frac{\pi}{2}- i\,\operatorname{sgn}(x)\,\arcosh\!\big(\sqrt{1+x^{2}}\big),\quad \operatorname{sgn}(0)=0.
  \end{equation*}
  特别地,当 $x\ge 0$ 时,有简洁形式
  \begin{equation*}
    \arccos(i x)=\frac{\pi}{2}- i\,\arcosh\!\big(\sqrt{1+x^{2}}\big).
  \end{equation*}
  上式的来龙去脉可以简述如下:令 $a=\arsinh x$(即 $\sinh a=x$),则
  \begin{equation*}
    \sin(i a)=i\sinh a=i x,\qquad \cos\Big(\tfrac{\pi}{2}-i a\Big)=\sin(i a)=i x,
  \end{equation*}
  因此满足 $\cos z=i x$ 的主值解是 $z=\tfrac{\pi}{2}-i\,\arsinh x$,即
  \begin{equation*}
    \arccos(i x)=\frac{\pi}{2}- i\,\arsinh x.
  \end{equation*}
  又因为 $\arcosh$ 的主值恒非负,且
  \begin{equation*}
    \arcosh\!\big(\sqrt{1+x^{2}}\big)=\ln\!\big(\sqrt{1+x^{2}}+|x|\big)=|\arsinh x|,
  \end{equation*}
  所以 $\arsinh x=\operatorname{sgn}(x)\,\arcosh\!\big(\sqrt{1+x^{2}}\big)$,从而得到上面的等式。
  其中 $\operatorname{sgn}(x)$ 是符号函数:$\operatorname{sgn}(x)=1\,(x>0)$,$\operatorname{sgn}(x)=0\,(x=0)$,$\operatorname{sgn}(x)=-1\,(x<0)$。
  
  导数从反函数求导公式出发就能搞定。因为 $\sinh(\arsinh x)=x$,两边对 $x$ 求导得到
  \begin{equation*}
    \cosh(\arsinh x) \cdot \frac{\mathrm{d}}{\mathrm{d}x}\arsinh x = 1.
  \end{equation*}
  注意到 $\cosh^{2} t - \sinh^{2} t = 1$,把 $t=\arsinh x$ 代进去有 $\cosh(\arsinh x)=\sqrt{x^{2}+1}$,于是
  \begin{equation*}
    \frac{\mathrm{d}}{\mathrm{d}x}\arsinh x = \frac{1}{\sqrt{x^{2}+1}}.
  \end{equation*}
  同理可以得到
  \begin{equation*}
    \frac{\mathrm{d}}{\mathrm{d}x}\arcosh x = \frac{1}{\sqrt{x-1}\sqrt{x+1}}, \qquad
    \frac{\mathrm{d}}{\mathrm{d}x}\artanh x = \frac{1}{1-x^{2}}.
  \end{equation*}
  它们的反过来积分公式就顺理成章:
  \begin{align}
    \int \frac{1}{\sqrt{x^{2}+1}}\,\mathrm{d}x &= \arsinh x + C, \\
    \int \frac{1}{\sqrt{x-1}\sqrt{x+1}}\,\mathrm{d}x &= \arcosh x + C, \\
    \int \frac{1}{1-x^{2}}\,\mathrm{d}x &= \artanh x + C.
  \end{align}
  遇到 $\sqrt{x^{2}+1}$ 这类根式时,直接用 $\sinh$ 换元往往比硬用三角函数更干脆,这些反双曲函数就是最后写答案时的“收尾动作”。
\end{example}

\begin{itemize}
  \item $(x^{n})^{\prime }=n x^{n-1},(e^{ax})^{\prime }=ae^{x},(a^{x})^{\prime }=(e^{x\ln a})^{\prime }=\ln a \cdot  a^{x}$
  \begin{problem}
    $(x^{x})^{\prime }=?,(x^{x^{x}})^{\prime }=?$
  \end{problem}
  \begin{solution}
    $x^{x}=e^{\ln(x^{x})}=e^{x\ln x},(x^{x})^{\prime }=e^{x \ln x}(\ln x+x\cdot \frac{1}{x})=x^{x}(\ln x+1)$
  \end{solution}
\end{itemize}

在继续看下去之前,确保你比自己的名字还熟悉以下函数的导数,因为这非常有利于你后面凑微分

\begin{itemize}
  \item 幂指对 \begin{equation}
    \begin{aligned}
      (x^{n})^{\prime }=n x^{n-1} &\quad (\ln x)^{\prime }=\frac{1}{x} \\
      (e^{ax})^{\prime }=ae^{ax} &\quad (a^{x})^{\prime }=(e^{x \ln a})^{\prime }=a^{x} \ln a \\
      (\sqrt{x})^{\prime }=\frac{1}{2 \sqrt{x}} & \quad 
    \end{aligned}
  \end{equation}

  \item 三角函数 
  \begin{equation}
    \begin{aligned}
    (\sin x)^{\prime }=\cos x &\quad (\cos x)^{\prime }=-\sin x\\
    (\tan x)^{\prime }=\frac{1}{\cos ^{2}x} & \quad (\cot x=\frac{1}{\tan x})^{\prime }=-\frac{1}{\sin ^{2}x}\\
    \end{aligned}
  \end{equation}
  \item 反三角
  \begin{equation}
    \begin{aligned}
    (\arctan x)^{\prime }=\frac{1}{1+x^{2}} \quad (\arcsin x)^{\prime }=\frac{1}{\sqrt{1-x^{2}}}  \\
    (\arccos x)'=(\frac{\pi}{2}-\arcsin x)'
=-\frac{1}{\sqrt{1-x^{2}}}    \end{aligned}
  \end{equation}
  
\end{itemize}
\section{洛必达法则}

若 $f(x_0)=0,g(x_0)= 0,g^{\prime }(x_0)=0$

则
\begin{equation}
  \lim_{x,\to x_0} \frac{f(x)}{g(x)}=\frac{f^{\prime }(x_0)}{g^{\prime }(x_0)}
\end{equation}

可以这样来简单理解:

在 $x_0$ 的足够小邻域内,任何函数 $f(x)$ 都可以近似成一条直线
,这条直线的斜率显然就是其导数 $f^{\prime }(x_0)$
,那么

\begin{equation}
  \begin{aligned}
  f(x)\approx f(x_0)+f^{\prime }(x_0)(x-x_0) \\
  g(x)\approx g(x_0)+g^{\prime }(x_0)(x-x_0)
  \end{aligned}
\end{equation}
而此时 $f(x_0)=0,g(x_0)=0$

显然 
\begin{equation}
  \lim_{x,\to x_0} \frac{f(x)}{g(x)}=\frac{f^{\prime }(x_0)(x-x_0)}{g^{\prime }(x_0)(x-x_0)}=\frac{f^{\prime }(x_0)}{g^{\prime }(x_0)}
\end{equation}


用洛必达我们可以验证一些极限,比如

\begin{example}
  \begin{equation}
    \begin{aligned}
    \lim_{x \to 0} \frac{\sin x }{x}=\lim_{x \to 0} \frac{(\sin x)^{\prime }}{x^{\prime }}=\lim_{x \to 0} \frac{\cos x}{1} \\
    \end{aligned}
  \end{equation}
\end{example}

\subsubsection{Stolz-Cesàro定理--离散化的洛必达法则}
Stolz定理是处理数列不定式极限的有力工具,可以看作是数列版本的洛必达法则。
\begin{itemize}
    \item $\frac{*}{\infty}$ 型: 若 $\{y_n\}$ 严格单增趋于 $+\infty$,且 $\lim_{n\to\infty} \frac{x_{n+1}-x_n}{y_{n+1}-y_n} = L$,则 $\lim_{n\to\infty} \frac{x_n}{y_n} = L$。
    \item $\frac{0}{0}$ 型: 若 $\{x_n\}, \{y_n\}$ 均趋于0,$\{y_n\}$ 严格单减,且 $\lim_{n\to\infty} \frac{x_{n+1}-x_n}{y_{n+1}-y_n} = L$,则 $\lim_{n\to\infty} \frac{x_n}{y_n} = L$。
\end{itemize}
\begin{problem}[CMC真题]
 设数列 $\{a_{n}\}$ 满足 $a_{1}>0,$ $a_{n+1}=a_{n}+\frac{1}{a_{n}},n\ge1.$ 证明:
$lim_{n\rightarrow\infty}\frac{a_{n}}{\sqrt{2n}}=1.$
\end{problem}
\begin{solution}
    显然 $\{a_n\}$ 是严格单增正数列。若 $\lim_{n\to\infty} a_n = A$ (有限),则 $A=A+\frac{1}{A}$,导出 $1/A=0$ 矛盾。故 $\lim_{n\to\infty} a_n = +\infty$。
    考虑 $\lim_{n\to\infty} \frac{a_n^2}{2n}$,这是一个 $\frac{\infty}{\infty}$ 型,适用Stolz定理。
    \begin{equation*}
        \lim_{n\to\infty} \frac{a_n^2}{2n} = \lim_{n\to\infty} \frac{a_{n+1}^2 - a_n^2}{2(n+1)-2n} = \frac{1}{2}\lim_{n\to\infty} \left( (a_n+\frac{1}{a_n})^2 - a_n^2 \right) = \frac{1}{2}\lim_{n\to\infty} (2 + \frac{1}{a_n^2}) = 1
    \end{equation*}
    因此 $\lim_{n\to\infty} \frac{a_n}{\sqrt{2n}} = 1$。
\end{solution}
\section{泰勒展开}

利用泰勒展开,可以将绝大部分性质良好的函数展开成级数的形式

\begin{equation}
  \begin{aligned}
  f(x)&=\sum_{n=1}^{\infty } \frac{f^{(n)}(x_0)}{n!} (x-x_0)^{n}\\
  &=f(x_0)+f'(x_0)(x-x_0)+\frac{f''(x_0)}{2} (x-x_0)^{2}+\frac{f'''(x_0)}{6}(x-x_0)^{3}+ \cdots \\
  \end{aligned} 
\end{equation}
如何理解此式?
简单的想法是,当我们的 $x$ 的取值始终在 $x_0$ 附近时, $(x-x_0)$ 是一个小量, $(x-x_0)^{n}$ 就是 $n$ 阶小量
在这个式子两边同时求导,然后忽略掉高阶小量,只留下量级最大的项
  \begin{equation}
  \begin{aligned}
  f(x)&=f(x_0)+f'(x_0)\Delta x+\frac{f''(x_0)}{2} (\Delta x)^{2}+\frac{f'''(x_0)}{6}(\Delta x)^{3}+ \cdots \approx f(x_0) \\
  f'(x)&=f'(x_0)+f''(x_0)\Delta x_0+\frac{f'''(x_0)}{2}(\Delta x)^{2}+ \cdots \approx f'(x_0)\\
  &\cdots
  \end{aligned} 
\end{equation}
上面的约等号在取 $\Delta x\to 0$ 时变成等号

这可以使我们相信 $f(x)=\sum_{n=1}^{\infty } \frac{f^{(n)}(x_0)}{n!} (x-x_0)^{n}$ 这个等式在 $f(x)$ 的任意阶导数都成立

下面列举一些常用导数的泰勒展开,其中 $O(x^{n})$ 表示还剩下一些项,这些项都是 $x^{n}$ 的高阶小量,我们加上 $O(x^{n})$ 用来提醒自己这样估算的误差量级大概有多大

\begin{itemize}
  \item $e^{x},$12345的阶乘 \begin{equation}
    e^{x}=\sum \frac{x^{n}}{n!} =1+x+\frac{x^{2}}{2}+\frac{x^{3}}{6}+ O(x^{3})
  \end{equation}
  \item 三角函数, $\sin x$ 奇数阶, $\cos x$ 偶数阶,
  \begin{equation}
    \sin x=
  \end{equation}
\end{itemize}
\section{常用求极限方法}
(1) 利用基本极限;
(2) 利用无穷小替换;
(3) 利用L' Hospital (洛必达)法则;
(4) 利用四抖。r (泰勒)公式;
(5) 利用导数的定义.
% \section{傅立叶级数}
% \section{傅立叶变换}
	\section{积分}
    \subsection{积分的定义}
    \subsection{什么叫积不出来?}
    \subsection{不定上下限积分的求导公式}
    \subsection{积分中值定理}
    \subsection{基本不定积分方法}
    \subsubsection{凑微分}
    \subsubsection{分部积分}
    \subsection{换元}
    \subsubsection{万能代换}
    \subsubsection{根式代换}
    \subsubsection{倒代换}
    \subsubsection{部分分式法}
    \subsubsection{留数法}
    \subsection{基本定积分方法}
    \subsubsection{区间再现}
    \subsubsection{点火}

\chapter{双元法}
\chapter{单元法}
\chapter{组合积分法}
\chapter{其他小方法}
\section{费曼求导法}
\section{循环法}
\section{先求递推,再求特值}
\section{rullani (傅汝兰尼)积分公式}

\appendix
\chapter{版本更新历史}

\datechange{2025/10/17}{编写第一章基础知识}
% \datechange{2025/10/05}{编写第一章}
% \begin{change}
%   \item 删除 \lstinline{mathpazo} 数学字体选项。
%   \item 添加邮箱命令 \lstinline{\mailto}。
%   \item 修改英文字体为 \lstinline{newtx} 系列,另外大型操作符号维持 cm 字体。
%   \item 中文字体改用 \lstinline{ctex} 宏包自动设置。
%   \item 删除 \lstinline{xeCJK} 字体设置,原因是不同系统字体不方便统一。
%   \item 定理换用 \lstinline{tcolorbox} 宏包定义,并基本维持原有的定理样式,优化显示效果,支持跨页;定理类名字重命名,如 etheorem 改为 theorem 等等。
%   \item 删去自定义的缩进命令 \lstinline{\Eindent}。
%   \item 添加参考文献宏包 \lstinline{natbib}。
%   \item 颜色名字重命名。
% \end{change}

\chapter{补充证明}

\section{重要极限的证明}\label{prf:ImtLim}
\begin{proof}[重要极限的证明]
  要严格证明 $\lim_{h \to 0} \frac{e^h - 1}{h} = 1$,我们不能使用 $(e^x)'=e^x$ 本身,否则会陷入循环论证。一个严谨的初等证明可以依赖于常数 $e$ 的另一个等价定义:
  \begin{equation}
    e = \lim_{k \to 0} (1+k)^{1/k}
  \end{equation}
  我们的证明步骤如下:
  \begin{enumerate}
    \item 设 $k = e^h - 1$。当 $h \to 0$ 时,显然 $e^h \to e^0 = 1$,因此 $k \to 0$。
    \item 从 $k = e^h - 1$ 中解出 $h$。我们得到 $e^h = 1+k$,两边取自然对数,得 $h = \ln(1+k)$。
    \item 将 $h$ 和 $k$ 的表达式代入原极限式中:
    \begin{equation}
      \lim_{h \to 0} \frac{e^h - 1}{h} = \lim_{k \to 0} \frac{k}{\ln(1+k)}
    \end{equation}
    \item 利用对数的性质 $\ln(a^b) = b\ln a$,对分母进行变形:
    \begin{equation}
      \lim_{k \to 0} \frac{1}{\frac{1}{k}\ln(1+k)} = \lim_{k \to 0} \frac{1}{\ln\left((1+k)^{1/k}\right)}
    \end{equation}
    \item 由于自然对数函数 $\ln(x)$ 是一个连续函数,我们可以将极限符号移到函数内部:
    \begin{equation}
      \frac{1}{\ln\left(\lim_{k \to 0}(1+k)^{1/k}\right)}
    \end{equation}
    \item 根据我们引用的 $e$ 的定义,括号内的极限正是 $e$。因此,上式变为:
    \begin{equation}
      \frac{1}{\ln(e)} = \frac{1}{1} = 1
    \end{equation}
  \end{enumerate}
  至此,证明完毕。这个证明虽然绕道使用了对数函数,但它依赖的是对数函数的连续性和代数性质,而非其导数,因此是有效的。
\end{proof}

\nocite{*}

\printbibliography[heading=bibintoc, title=\ebibname]
参考文献尚未编排,此处使用的是模板的默认参考文献,作废
% \appendix

\chapter{积分表}


% \section{求和算子与描述统计量}




\end{document}
