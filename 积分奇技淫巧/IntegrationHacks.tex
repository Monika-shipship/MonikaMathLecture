\documentclass[lang=cn,newtx,10pt,scheme=chinese]{elegantbook}

\title{积分的奇技淫巧}
\subtitle{Integration Hacks}

\author{Monika}
% \institute{Elegant\LaTeX{} Program}
\date{\today}
\version{0.1}
% \bioinfo{自定义}{信息}

% \extrainfo{注意:本模板自 2023 年 1 月 1 日开始,不再更新和维护!}

\setcounter{tocdepth}{3}

\logo{logo-blue.png}
\cover{cover.jpg}

% 本文档命令
\usepackage{array}
\newcommand{\ccr}[1]{\makecell{{\color{#1}\rule{1cm}{1cm}}}}

\renewcommand{\textbf}[1]{\text{\heiti #1}}

% 修改标题页的橙色带
\definecolor{customcolor}{RGB}{32,178,170}
\colorlet{coverlinecolor}{customcolor}
\usepackage{cprotect}
\usepackage{tikz}
\usepackage{float}
\usepackage{soul}
\usepackage{arcs}
\usetikzlibrary{arrows.meta,3d}

\addbibresource[location=local]{reference.bib} % 参考文献,不要删除
\definecolor{lightyellow}{rgb}{1, 1, 0.8} % 定义一个淡黄色
\sethlcolor{lightyellow} % 设置为高亮颜色
\begin{document}

\maketitle
\frontmatter

\tableofcontents

\chapter{绪论}
本书是是 LZU 数学协会举办的数学讲座的讲义,将会讲授一些在课不上不会讲授,但在某些情况有奇效的积分技巧,近年来 cmc 的试题中偶尔会出现此类题目

笔者水平有限,若有不足缺漏之处恳请读者更正,或在 \href{https://github.com/Monika-shipship/MonikaMathLecture}{Github仓库处(点击即可跳转)} 提交PR以更正
\mainmatter

\chapter{基础知识}
在本章列举出关于微积分最基础的知识,读者应做到在看到本章的所有题目时能瞬间反应出答案来
\section{极限}
所谓的极限,就是 Approach ,不断地逼进一个值,但是始终不会到达

用数学语言来讲就是
\begin{equation}
  \lim_{x \to a} f(x)=A \iff \forall \epsilon>0,\exists \delta>0, \text{使得当}x \in \left[ a-\epsilon,a+\epsilon \right], \left\vert f(x) -A\right\vert <\epsilon
\end{equation}
如果函数是多元函数,则定义为从任意路径趋近于该点的值都相同,注意是任意路径而不是任意角度,取所有斜率的直线并不能穷尽所有路径,比如下面这个
\begin{problem}
    $\lim_{(x,y) \to (0,0)} \frac{xy}{x+y} $ 是否存在?
  \end{problem}
  如果你取所有斜率的直线,比如 $y=kx$,而误以为直线能穷尽所有路径,就会得出极限为零的错误答案 $\lim_{(x,y) \to (0,0)} f(x,y)=\lim \frac{(kx^{2})}{(1+k)x}=\lim_{x \to 0} \frac{k}{(1+k)}x=0$
  \begin{solution}
    不存在,取路径 $y=-x+x^{2},\lim_{(x,y) \to (0,0)} \frac{x(-x-x^{2})}{x^{2}}=\lim_{x \to 0}(-1-x)=-1$\
    这条路径得到的值和 $y=kx$ 路径得到的值不同,所以极限不存在
  \end{solution}
\section{导数}
我们使用最朴素的导数理解方式,自变量发生微小变化后,因变量会随之发生一个微小变化,这两个变化的比值就是导数,也是斜率

\begin{equation}
  \frac{\mathrm{d}y}{\mathrm{d}x} \big|_{x_0}\equiv \lim_{\Delta x \to 0} \frac{\Delta y}{\Delta x}=\lim_{x \to x_0} \frac{y(x)-y(x_0)}{x-x_0}
\end{equation}
利用这个定义不难算出一些初等函数的导数,这里只以 $x^{n}$ 和 $\sin x$ 的导数为例
\begin{example}
  \begin{equation}
    (x^{n})^{\prime }=\frac{\Delta x^{n}}{\Delta x}=\frac{(x+\Delta x)^{n}-x^{n}}{\Delta x}=\frac{x^{n}+}{1}
  \end{equation}
\end{example}
\begin{itemize}
  \item $(x^{n})^{\prime }=n x^{n-1},(e^{ax})^{\prime }=ae^{x},(a^{x})^{\prime }=(e^{x\ln a})^{\prime }=\ln a \cdot  a^{x}$
  \begin{problem}
    $(x^{x})^{\prime }=?,(x^{x^{x}})^{\prime }=?$
  \end{problem}
  \begin{solution}
    $x^{x}=e^{\ln(x^{x})}=e^{x\ln x},(x^{x})^{\prime }=e^{x \ln x}()$
  \end{solution}
\end{itemize}
\section{洛必达法则}
\section{泰勒展开}
\section{常用求极限方法}
(1) 利用基本极限;
(2) 利用无穷小替换;
(3) 利用L' Hospital (洛必达)法则;
(4) 利用四抖。r (泰勒)公式;
(5) 利用导数的定义.
% \section{傅立叶级数}
% \section{傅立叶变换}
	\section{积分}
    \subsection{积分的定义}
    \subsection{什么叫积不出来?}
    \subsection{不定上下限积分的求导公式}
    \subsection{积分中值定理}
    \subsection{基本不定积分方法}
    \subsubsection{凑微分}
    \subsubsection{分部积分}
    \subsection{换元}
    \subsubsection{万能代换}
    \subsubsection{根式代换}
    \subsubsection{倒代换}
    \subsubsection{部分分式法}
    \subsubsection{留数法}
    \subsection{基本定积分方法}
    \subsubsection{区间再现}
    \subsubsection{点火}

\chapter{双元法}
\chapter{单元法}
\chapter{组合积分法}
\chapter{其他小方法}
\section{费曼求导法}
\section{循环法}
\section{先求递推,再求特值}
\section{rullani (傅汝兰尼)积分公式}


\chapter{版本更新历史}

\datechange{2025/10/07}{编写第一章}
% \datechange{2025/10/05}{编写第一章}
% \begin{change}
%   \item 删除 \lstinline{mathpazo} 数学字体选项。
%   \item 添加邮箱命令 \lstinline{\mailto}。
%   \item 修改英文字体为 \lstinline{newtx} 系列,另外大型操作符号维持 cm 字体。
%   \item 中文字体改用 \lstinline{ctex} 宏包自动设置。
%   \item 删除 \lstinline{xeCJK} 字体设置,原因是不同系统字体不方便统一。
%   \item 定理换用 \lstinline{tcolorbox} 宏包定义,并基本维持原有的定理样式,优化显示效果,支持跨页;定理类名字重命名,如 etheorem 改为 theorem 等等。
%   \item 删去自定义的缩进命令 \lstinline{\Eindent}。
%   \item 添加参考文献宏包 \lstinline{natbib}。
%   \item 颜色名字重命名。
% \end{change}

\nocite{*}

\printbibliography[heading=bibintoc, title=\ebibname]
参考文献尚未编排,此处使用的是模板的默认参考文献,作废
\appendix

\chapter{积分表}


% \section{求和算子与描述统计量}




\end{document}
