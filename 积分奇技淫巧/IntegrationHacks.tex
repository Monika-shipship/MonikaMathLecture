\documentclass[lang=cn,newtx,10pt,scheme=chinese]{elegantbook}

\title{积分的奇技淫巧}
\subtitle{Integration Hacks}

\author{Monika}
% \institute{Elegant\LaTeX{} Program}
\date{\today}
\version{0.1}
% \bioinfo{自定义}{信息}

% \extrainfo{注意:本模板自 2023 年 1 月 1 日开始,不再更新和维护!}

\setcounter{tocdepth}{3}

\logo{logo-blue.png}
\cover{cover.jpg}

% 本文档命令
\usepackage{array}
\newcommand{\ccr}[1]{\makecell{{\color{#1}\rule{1cm}{1cm}}}}

\renewcommand{\textbf}[1]{\text{\heiti #1}}

% 修改标题页的橙色带
\definecolor{customcolor}{RGB}{32,178,170}
\colorlet{coverlinecolor}{customcolor}
\usepackage{cprotect}
\usepackage{tikz}
\usepackage{float}
\usepackage{soul}
\usepackage{arcs}
\usetikzlibrary{arrows.meta,3d}

\addbibresource[location=local]{reference.bib} % 参考文献,不要删除
\definecolor{lightyellow}{rgb}{1, 1, 0.8} % 定义一个淡黄色
\sethlcolor{lightyellow} % 设置为高亮颜色
\begin{document}

\maketitle
\frontmatter

\tableofcontents

\chapter{绪论}
本书是是 LZU 数学协会举办的数学讲座的讲义,将会讲授一些在课不上不会讲授,但在某些情况有奇效的积分技巧,近年来 cmc 的试题中偶尔会出现此类题目

笔者水平有限,若有不足缺漏之处恳请读者更正,或在 \href{Github仓库处}{}提交PR以更正
\mainmatter


	

\chapter{版本更新历史}

\datechange{2025/10/07}{编写第一章}
% \datechange{2025/10/05}{编写第一章}
% \begin{change}
%   \item 删除 \lstinline{mathpazo} 数学字体选项。
%   \item 添加邮箱命令 \lstinline{\mailto}。
%   \item 修改英文字体为 \lstinline{newtx} 系列,另外大型操作符号维持 cm 字体。
%   \item 中文字体改用 \lstinline{ctex} 宏包自动设置。
%   \item 删除 \lstinline{xeCJK} 字体设置,原因是不同系统字体不方便统一。
%   \item 定理换用 \lstinline{tcolorbox} 宏包定义,并基本维持原有的定理样式,优化显示效果,支持跨页;定理类名字重命名,如 etheorem 改为 theorem 等等。
%   \item 删去自定义的缩进命令 \lstinline{\Eindent}。
%   \item 添加参考文献宏包 \lstinline{natbib}。
%   \item 颜色名字重命名。
% \end{change}

\nocite{*}

\printbibliography[heading=bibintoc, title=\ebibname]
参考文献尚未编排,此处使用的是模板的默认参考文献,作废
\appendix

\chapter{基本数学工具}


% \section{求和算子与描述统计量}




\end{document}
